
\addchap{Vorbemerkungen}

\section*{Über dieses Buch}
\label{sec:ueberdiesesbuch}

\ldots

Ein weiteres Desideratum, das dieses Buch einlöst, ist das nach einem konzisen Abriss der wesentlichen inhaltlichen Punkte aus EGBD.
Alle Kapitel in GAG beginnen mit einer Kurzdarstellung des Themengebiets und der zugehörigen Analysewerkzeuge (einschließlich Notationskonventionen und Strukturformate), die in EGBD eingeführt werden.
Aus den verkürzten Darstellungen hier geht oft nicht die volle Motivation hervor, aus der heraus verschiedene Analysen vertreten werden, sondern sie werden im Ernstfall schlicht als scheinbare Wahrheiten eingeführt.
Der Eindruck der Willkürlichkeit oder \textit{Analyse um der Analyse willen} kann daher wohl bedauerlicherweise nicht immer vermieden werden, aber GAG versteht sich eben vor allem als Übungsbuch für diejenigen, die EGBD bereits durchgearbeitet haben, und die daher bereits verstanden haben, dass alle Analysen an konkretem sprachlichen Material begründet wurden.

\section*{Benutzung dieses Buchs}
\label{sec:benutzungdiesesbuchs}

\ldots

Im Sinne der obigen Erläuterungen wäre der ungünstigste Fall ein Einsatz dieses Buches als Grammatikfibel zum Pauken für irgendwelche Klausuren \textit{ohne} die vorherige gründliche Lektüre von EGBD und\slash oder anderen Einführungen.

\section*{Danksagungen}
\label{sec:danksagungen}

\section*{Webseite}
\label{sec:webseite}

Es gibt eine Webseite zu diesem Buch und \textit{Einführung in die grammatische Beschreibung des Deutschen} mit zusätzlichen Materialien und Diskussionen über Grammatik:

\begin{center}
  \url{http://grammatick.de/}
\end{center}

