\chapter{Phonetik}
\label{sec:phonetik}

\section{Aufgabe und Methode}
\label{sec:phonetik:aufgabeundmethoden}

In \EGBD wird die artikulatorische Phonetik nicht eingeführt, wie man dies in einer ausgewiesenen Phonetik"=Einführung tun würde.
Man würde dort das Gehör und die Artikulationsfähigkeit der Lernenden trainieren, um genau das in IPA"=Notation aufzuschreiben, was gehört wurde.
Dieses Vorgehen ist bei der Beschreibung von Sprachen und Dialekten von großer Bedeutung, und oft werden Messinstrumente hinzugezogen, um das Gehörte noch präziser notieren zu können.
In \EGBD wird hingegen eine Transkription der in Deutschland gesprochenen Standardaussprache eingeführt, also sozusagen eine Übersetzung von orthographischen Formen in phonetische Beschreibungen.

Dass dies so ist, hat zunächst einen praktischen Grund:
Für eine ordentliche Ausbildung in Phonetik ist sowohl im Buch als auch in den meisten germanistischen Studiengängen kein Raum.
Man muss sich vergegenwärtigen, dass die Ausbildung in richtiger Phonetik Jahre dauert und viel Übung erfordert, da die zu hörenden bzw.\ zu messenden phonetischen Sachverhalte oft subtil sind.
Der inhaltliche Grund für das Vorgehen ist, dass die Standardaussprache eine besondere Bedeutung im sprachlichen Leben, in der Schule und im Studium hat.
Insbesondere gelten die folgenden Punkte.

\begin{enumerate}
  \item Die Standardorthographie wird i.\,d.\,R.\ mit Bezug auf die Phonologie des Standards beschrieben, die wiederum von der Standardaussprache abhängt.
  \item Wer nicht selber eine Ausbildung in Dialektologie hat versteht Dialekte (und Soziolekte, Kiezsprachen usw.) meist gut im Kontrast zum Standard.
  \item Im schulischen Deutschunterricht muss unbedingt der Standard unterrichtet werden, denn er genießt in vielen Situationen besonderes Prestige oder wird sogar erwartet (vgl.\ \citealt[7]{KrechEa2009}).
  \item Damit ist in der mündlichen Kommunikation die Standardaussprache ein zentrales und meist sofort (nach wenigen Wörtern) erkennbares Merkmal bildungssprachlicher Kompetenz (oder Inkompetenz).
  \item Studierende -- und damit angehende Lehrpersonen -- haben oft keine präzise Vorstellung, wie genau die Standardaussprache kodifiziert ist.
\end{enumerate}

Vor diesem Hintergrund bestehen die Übungen zur Phonetik in diesem Buch (wie bereits in \EGBD) darin, dass Sätze in die IPA-Notation des Standards übersetzt werden.
Die Frage, die sich dabei aufdrängt, ist die, wo das Wissen um den Standard herkommt bzw.\ wo es kodifiziert wurde.
Das ist insbesondere im Fall der Standardaussprache eine delikate Frage.

\citet{DeppermannEa2013,Kleiner2014}

\section{Abweichungen vom De Gruyter-Handbuch in EGBD}
\label{sec:phonetik:abweichungenvomdegruyterhandbuchinegbd}

\section{Unterschiede zwischen EGBD und Grundriss}
\label{sec:phonetik:unterschiedezwischenegbdundgrundriss}

\section{Phonetische Analysen}
\label{sec:phonetik:phonetischeanalysen}

\subsection{(Titel der Aufgabe)}

\subsubsection{Aufgabe}

\subsubsection{Lösung nach EGBD}

\subsubsection{Abweichungen und Ergänzungen im Grundriss}


