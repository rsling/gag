\chapter{Phonetik}
\label{sec:phonetik}

\Voraussetzungen{Für das gesamte Kapitel sind gute Kenntnisse in der Definition und der Transkription der Standardaussprache unabdinglich (\EGBDRef{Phonetik}).
Es wird empfohlen, das Audiomaterial auf der Webseite zum Buch durchzuarbeiten.
Für Zweifelsfälle sollte \citet{KrechEa2009} zur Hand genommen werden.
}

\section{Aufgaben der Phonetik}
\label{sec:phonetik:aufgabenderphonetik}

In EGBD wird die artikulatorische Phonetik nicht eingeführt, wie man dies in einer ausgewiesenen Phonetik"=Einführung tun würde.
Man würde dort das Gehör und die Artikulationsfähigkeit der Lernenden trainieren, um genau das in IPA"=Notation aufzuschreiben, was gehört wurde.
Dieses Vorgehen ist bei der Beschreibung von Sprachen und Dialekten von großer Bedeutung, und oft werden Messinstrumente hinzugezogen, um das Gehörte noch präziser notieren zu können.
In EGBD wird hingegen eine Transkription der in Deutschland gesprochenen Standardaussprache eingeführt, also sozusagen eine Übersetzung von orthographischen Formen in phonetische Beschreibungen.

Dass dies so ist, hat zunächst einen praktischen Grund:
Für eine ordentliche Ausbildung in Phonetik ist sowohl im Buch als auch in den meisten germanistischen Studiengängen kein Raum.
Man muss sich vergegenwärtigen, dass die Ausbildung in richtiger Phonetik Jahre dauert und viel Übung erfordert, da die zu hörenden bzw.\ zu messenden phonetischen Sachverhalte oft subtil sind.
Der inhaltliche Grund für das Vorgehen ist, dass die Standardaussprache eine besondere Bedeutung im sprachlichen Leben, in der Schule und im Studium hat.
Insbesondere gelten die folgenden Punkte.

\begin{enumerate}
  \item Die Standardorthographie wird i.\,d.\,R.\ mit Bezug auf die Phonologie des Standards beschrieben, die wiederum von der Standardaussprache abhängt.
  \item Wer nicht selber eine Ausbildung in Dialektologie hat versteht Dialekte (und Soziolekte, Kiezsprachen usw.) meist gut im Kontrast zum Standard.
  \item Im schulischen Deutschunterricht muss unbedingt der Standard unterrichtet werden, denn er genießt in vielen Situationen besonderes Prestige oder wird sogar erwartet (vgl.\ \citealt[7]{KrechEa2009}).
  \item Damit ist in der mündlichen Kommunikation die Standardaussprache ein zentrales und meist sofort (nach wenigen Wörtern) erkennbares Merkmal bildungssprachlicher Kompetenz (oder Inkompetenz).
  \item Studierende -- und damit angehende Lehrpersonen -- haben oft keine präzise Vorstellung, wie genau die Standardaussprache kodifiziert ist.
\end{enumerate}

Vor diesem Hintergrund bestehen die Übungen zur Phonetik in diesem Buch (wie bereits in EGBD) darin, dass Sätze in die IPA-Notation des Standards übersetzt werden.
Die Frage, die sich dabei aufdrängt, ist die, wo das Wissen um den Standard herkommt bzw.\ wo es kodifiziert wurde.
Das ist insbesondere im Fall der Standardaussprache eine delikate Frage.

\citet{DeppermannEa2013,Kleiner2014}

\vspace{2\baselineskip}
\Text{Weltraum}{%
\label{text:weltraum}

\subsection*{Einleitung}

Der Weltraum bezeichnet den Raum zwischen Himmelskörpern.
Die Atmosphären von festen und gasförmigen Himmelskörpern (wie Sternen und Planeten) haben keine feste Grenze nach oben, sondern werden mit zunehmendem Abstand zum Himmelskörper allmählich immer dünner.
Ab einer bestimmten Höhe spricht man vom Beginn des Weltraums.

Im Weltraum herrscht ein Hochvakuum mit niedriger Teilchendichte.
Er ist aber kein leerer Raum, sondern enthält Gase, kosmischen Staub und Elementarteilchen (Neutrinos, kosmische Strahlung, Partikel), außerdem elektrische und magnetische Felder, Gravitationsfelder und elektromagnetische Wellen (Photonen).
Das fast vollständige Vakuum im Weltraum macht ihn außerordentlich durchsichtig und erlaubt die Beobachtung extrem entfernter Objekte, etwa anderer Galaxien.
Jedoch können Nebel aus interstellarer Materie die Sicht auf dahinterliegende Objekte auch stark behindern.

Der Begriff des Weltraums ist nicht gleichzusetzen mit dem Weltall, welches eine eingedeutschte Bezeichnung für das Universum insgesamt ist und somit alles, also auch die Sterne und Planeten selbst, mit einschließt.
Dennoch wird das deutsche Wort \textit{Weltall} oder \textit{All} umgangssprachlich (eigentlich inkorrekt) mit der Bedeutung \textit{Weltraum} verwendet.

Die Erforschung des Weltraums wird \textit{Weltraumforschung} genannt.
Reisen oder Transporte in oder durch den Weltraum werden als Raumfahrt bezeichnet.

\subsection*{Beginn des Weltraums}

Die Übergangszone zwischen der Erdatmosphäre und dem Weltraum, mit der Mondsichel im Hintergrund.
Der Übergang zwischen der Erdatmosphäre und dem Weltraum ist fließend.
Die Fédération Aéronautique Internationale (FAI) definiert die Grenze zum Weltraum bei 100 Kilometern Höhe über dem Meeresspiegel, der Kármán-Linie.
In dieser Höhe ist die Geschwindigkeit, die benötigt wird, um Auftrieb zum Fliegen zu erhalten, gleich hoch wie die Umlaufgeschwindigkeit eines Satelliten, so dass man oberhalb dieser Linie nicht mehr sinnvoll von Luftfahrt sprechen kann.
Auch die NASA schließt sich der 100-Kilometer-Definition an.
Davon abweichend definiert die US Air Force bereits die Höhe von 50 Meilen (circa 80 km) als Beginn des Weltraums.
Beide als Grenzen vorgeschlagenen Höhen liegen in der Hochatmosphäre.
Eine völkerrechtlich verbindliche Höhengrenze zum Weltraum gibt es nicht.

Eine andere Höhendefinition, die diskutiert wird, ist die niedrigstmögliche Perigäumshöhe eines Erdsatelliten, da die dünne Atmosphäre auch oberhalb von 100 Kilometern noch eine nicht zu vernachlässigende Bremswirkung hat.
Bei einem die Erde elliptisch umkreisenden Raumflugkörper mit Antrieb liegt die niedrigstmögliche Perigäumshöhe bei etwa 130 Kilometern.
Bei einem Raumflugkörper ohne Antrieb liegt sie bei ungefähr 150 Kilometern.
Aber selbst in 400 Kilometern, der Flughöhe der Internationalen Raumstation, ist noch eine Bremswirkung der Atmosphäre spürbar, durch die die ISS ständig leicht an Höhe verliert und immer wieder von angedockten Raumschiffen auf eine höhere Umlaufbahn zurückgeschoben werden muss.

Die Kármán-Linie der Venus befindet sich bei ungefähr 250 Kilometern Höhe, die des Mars bei etwa 80 Kilometern.
Bei Himmelskörpern, die keine oder fast keine Atmosphäre haben, wie etwa dem Merkur, dem Erdmond oder Asteroiden, beginnt der Weltraum direkt an der Oberfläche des Körpers.

Beim Wiedereintritt von Raumflugkörpern in die Atmosphäre wird für die Berechnung der Flugbahn eine Wiedereintrittshöhe so festgelegt, dass bis zum Wiedereintrittspunkt der Einfluss der Atmosphäre praktisch vernachlässigbar ist; ab diesem Punkt muss er einkalkuliert werden.
Üblicherweise ist die Wiedereintrittshöhe gleich oder höher der Kármán-Linie.
Die NASA verwendet bei der Erde als Wiedereintrittshöhe den Wert von 400.000 Fuß (ca. 122 Kilometer).

\noindent{\footnotesize Quelle: \textit{Weltraum}, \url{https://de.wikipedia.org/wiki/Weltraum} (Auszug, bearbeitet)}
}


\Uebung{Transkription}{%
Transkribieren Sie Text~\ref{text:weltraum} im bundesdeutschen Standard in IPA.
Silbengrenzen und Akzente (Betonungen) müssen noch nicht notiert werden.
}

\paragraph*{Teillösung}

Die Transkription wird interlinear mit den Originalsätzen gegeben.

\begin{exe}
  \ex \gll Der Weltraum bezeichnet den Raum zwischen Himmelskörpern.\\
  de͡ɐ vɛltʁa͡ɔm bət͡sa͡ɛçnət deːn ʁa͡ɔm t͡svɪʃən hɪməlskœ͡əpɐn\\
  \ex \gll Die Atmosphären von festen und gasförmigen Himmelskörpern (wie Sternen und Planeten) haben keine feste Grenze nach oben, sondern werden mit zunehmendem Abstand zum Himmelskörper allmählich immer dünner.\\
  diː atmosfɛːrən vɔn fɛstən ʊnt gaːsfœ͡əmɪgən hɪməlskœ͡əpɐn viː ʃtɛ͡ənən ʔʊnt planeːtən haːbən ka͡ɛnə fɛstə gʁɛnt͡sə naːχ ʔoːbən zɔndɐn vɛ͡ədən mɪt t͡suːneːməndəm ʔapʃtant t͡sʊm hɪməlskœ͡əpɐ almɛːlɪç ʔɪmɐ dʏnɐ\\
  \ex \gll Ab einer bestimmten Höhe spricht man vom Beginn des Weltraums.\\
  ʔap ʔa͡ɛnɐ bəʃtɪmtən høːə ʃpʁɪçt man fɔm bəgɪn dəs vɛltʁa͡ɔms\\
  \ex \gll Im Weltraum herrscht ein Hochvakuum mit niedriger Teilchendichte.\\
  ʔɪm vɛltʁa͡ɔm hɛ͡əʃt ʔa͡ɛn hoːχvaːkuʔʊm mɪt niːdʁɪgɐ ta͡ɛlçəndɪçtə\\
  \ex \gll Er ist aber kein leerer Raum, sondern enthält Gase, kosmischen Staub und Elementarteilchen (Neutrinos, kosmische Strahlung, Partikel), außerdem elektrische und magnetische Felder, Gravitationsfelder und elektromagnetische Wellen (Photonen).\\
  ʔe͡ɐ ʔɪst ʔaːbɐ ka͡ɛn leːrɐ ʁa͡ɔm zɔndɐn ʔɛnthɛlt gaːzə kɔsmɪʃən ʃta͡ɔp ʔʊnt ʔɛləmɛnta͡ɐta͡ɛlçən nɔ͡œtʁiːnos kɔsmɪʃə ʃtʁaːlʊŋ pa͡ətikəl ʔa͡ɔsɐdeːm ʔelɛktʁɪʃə ʔʊnt magneːtɪʃə fɛldɐ gʁavitat͡sioːnsfɛldɐ ʔʊnt ʔelɛktʁomagneːtɪʃə vɛlən fotoːnən\\
  \ex \gll Das fast vollständige Vakuum im Weltraum macht ihn außerordentlich durchsichtig und erlaubt die Beobachtung extrem entfernter Objekte, etwa anderer Galaxien.\\
  das fast fɔlʃtɛndɪgə vaːkuʔʊm ʔɪm vɛltʁa͡ɔm maχt ʔiːn ʔa͡ɔsɐʔɔ͡ədəntlɪç dʊ͡əçzɪçtɪç ʔʊnt ʔɛ͡əla͡ɔpt diː bəʔoːbaχtʊŋ ʔɛkstʁeːm ʔɛntfɛ͡əntɐ ʔɔpjɛktə ʔɛtvaː ʔandəʁɐ galaksiːən\\
  \ex \gll Jedoch können Nebel aus interstellarer Materie die Sicht auf dahinterliegende Objekte auch stark behindern.\\
  jedɔχ kœnən neːbəl ʔa͡ɔs ʔɪntɐstɛlaːʁɐ mateːʁiə diː zɪçt ʔa͡ɔf dahɪntɐliːgəndə ʔɔpjɛktə ʔa͡ɔχ ʃta͡ək bəhɪndɐn\\
\end{exe}
