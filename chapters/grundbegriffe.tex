\chapter{Grundbegriffe}

\section{Allgemeine Grundbegriffe}

\paragraph*{Struktur, Relation, Funktion}

\paragraph*{Typen- und Tokenhäufigkeit}

\paragraph*{Kern und Peripherie}

\paragraph*{Wortbegriffe}


\section{Phonetische und phonologische Grundbegriffe}

\paragraph*{Transkription und Transliteration}

\paragraph*{Segmente und Merkmale}

\paragraph*{Strukturbedingungen und Anpassungsprozesse}

\paragraph*{Silben und Silbifizierung}


\section{Morphologische Grundbegriffe}

\paragraph*{Morphe und Markierungsfunktion}

\paragraph*{Stämme, Affixe und Köpfe}

\paragraph*{Flexion und Wortbildung}

\paragraph*{Flexionskategorien der Verben und Nomina}

\paragraph*{Stark, schwach, gemischt}


\section{Syntaktische Grundbegriffe}

\paragraph*{Kongruenz, Rektion und Valenz}

\paragraph*{Konstituenten}

\paragraph*{Satzglieder}

\paragraph*{Phrasenstruktur und Phrasenschemata}

\paragraph*{Köpfe}

\paragraph*{Semantische Rollen}

\paragraph*{Prädikate, Subjekte, Objekte}


\section{Anmerkungen zu den Wortklassen}

\paragraph*{Filtermethode}

\paragraph*{Probleme der Klassifikation und Prototypie}

\paragraph*{Braucht man Wortklassen?}


