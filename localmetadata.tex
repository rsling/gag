\title{Grammatische Analysen für das Germanistik\-studium}
\subtitle{}
\BackTitle{Grammatische Analysen für das Germanistik\-studium}
\BackBody{\begin{sloppypar}

\noindent Dieses Buch füllt eine Lücke im Büchermarkt für das germanistische BA-Studium.
Es gibt zahlreiche Einführungen in die Linguistik und Grammatik des Deutschen, aber Lehrbücher mit ausführlichen Übungen anhand von realistischem sprachlichen Material sind rar.
In diesem Buch werden belegte Sätze und Textstücke konzeptuell vollständig phonologisch, morphologisch, syntaktisch und graphematisch analysiert.
Als Beschreibungsrahmen wird die \textit{Einführun in die grammatische Beschreibung des Deutschen} von Roland Schäfer gewählt.
Zu jeder Analyse werden aber Abweichungen und Ergänzungen anhand von Peter Eisenbergs \textit{Grundriss der deutschen Grammatik} diskutiert.

\vspace{0.5\baselineskip}

\noindent\textbf{Roland Schäfer} ist Germanist und Linguist.
Er hat an der Phi\-lipps-\-Universität Marburg studiert und war wissenschaftlicher Mitarbeiter an der Georg-August Universität Göttingen und der Freien Universität Berlin.
Er hat Professuren in Göttingen (2011\slash 2012) und an der Freien Universität Berlin (2016 und seit 2018) vertreten.
Er hat 2018 eine Habilitationsschrift zum Thema \textit{Probabilistic German Morphosyntax} (eine Analyse von sogenannten \textit{Zweifelsfällen} im Rahmen der probabilistischen Grammatik) vorgelegt.
Seine aktuellen Forschungsschwerpunkte sind die probabilistische Morphosyntax und Graphematik des Deutschen, empirische und statistische Verfahren, Fachdidaktik und Lehramtsausbildung sowie die Korpuserstellung.
\end{sloppypar}

\vspace{0.5\baselineskip}

\noindent\textbf{Ulrike Sayatz} ist Germanistin und Linguistin.
Sie ist sehr gut usw.
}

\dedication{}

\typesetter{Roland Schäfer}

\proofreader{}

\author{Roland Schäfer\lastand Ulrike Sayatz}

\BookDOI{}
\renewcommand{\lsISBNdigital}{}
\renewcommand{\lsISBNhardcover}{}
\renewcommand{\lsISBNsoftcover}{}
\renewcommand{\lsISBNsoftcoverus}{}
\renewcommand{\lsSeries}{tbls}
\renewcommand{\lsSeriesNumber}{0}
\renewcommand{\lsURL}{}

 
 
 
 
  
