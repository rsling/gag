% By LSP.
%\renewbibmacro*{index:name}[5]{%
%  \usebibmacro{index:entry}{#1}
%    {\iffieldundef{usera}{}{\thefield{usera}\actualoperator}\mkbibindexname{#2}{#3}{#4}{#5}}}


% By LSP.
\makeatletter
\def\blx@maxline{77}
\makeatother


\def\justifying{%
  \rightskip=0pt
  \spaceskip=0pt
  \xspaceskip=0pt
  \relax
}

% Mark for proof-reading/revision.
\newcommand{\Dirty}{\marginpar{\hl{DIRTY!}}}


% Fix line spacing in list environmens.
\setlist{noitemsep}


% Correct hyperref colors. The original ones would give you eye cancer.
\hypersetup{
  linkbordercolor  = {white}
%  , linkcolor        = {lsMidDarkBlue}
%  , anchorcolor      = {lsMidWine}
%  , citecolor        = {lsDarkGreenOne}
%  , menucolor        = {lsMidDarkBlue}
%  , urlcolor         = {lsDarkOrange}
%    , filecolor       = {}
%    , runcolor        = {}
}


% Use a better mono font, ideal for code.
% https://github.com/chrissimpkins/codeface/tree/master/fonts/inconsolata-g
% \setmonofont{Inconsolata-g}


% Use a math font that actually works! Requires unicode-math paackage.
% https://github.com/khaledhosny/libertinus
\setmathfont[Scale=MatchUppercase]{Libertinus Math}


\newenvironment{listLileveli}{\begin{enumerate}}{\end{enumerate}}
\newenvironment{listLilevelii}{\begin{enumerate}}{\end{enumerate}}
\newenvironment{listLileveliii}{\begin{enumerate}}{\end{enumerate}}
\newenvironment{listLileveliv}{\begin{enumerate}}{\end{enumerate}}
\newenvironment{listLiileveli}{\begin{itemize}}{\end{itemize}}
\newenvironment{listLiilevelii}{\begin{itemize}}{\end{itemize}}
\newenvironment{listLiileveliii}{\begin{itemize}}{\end{itemize}}
\newenvironment{listLiileveliv}{\begin{itemize}}{\end{itemize}} 

% define own variables for hacks
\newlength{\myBuffer}

% formatting
\newcommand{\Lf}{
  \setlength{\itemsep}{1pt}
  \setlength{\parskip}{0pt}
  \setlength{\parsep}{0pt}
}

% Feinsatz 

\newcommand{\Stretch}[1][1]{\vspace{#1\baselineskip}}
\newcommand{\Unstretch}[1][0.5]{\vspace{-#1\baselineskip}}
\newcommand{\Hardstretch}[1][1]{\vspace*{#1\baselineskip}}
\newcommand{\Enl}[1][1]{\enlargethispage{#1\baselineskip}}
\newcommand{\Np}{\newpage}

% Uncomment to switch off
%\renewcommand{\Stretch}[1][1]{}
%\renewcommand{\Unstretch}[1][0.5]{}
%\renewcommand{\Hardstretch}[1][1]{}
%\renewcommand{\Enl}[1][1]{}
%\renewcommand{\Np}{}

\newcommand{\EGBD}{Einführung in die grammatische Beschreibung des Deutschen{}}
\newcommand{\GR}[1][]{\ifthenelse{\equal{#1}{Gen}}{Grundrisses der deutschen Grammatik{}}{Grundriss der deutschen Grammatik{}}}

% all sorts of shortcuts
\newcommand{\HStrut}[1]{\rule{0pt}{#1pt}}
\newcommand{\VStrut}[1]{\rule{#1pt}{0pt}}
\newcommand{\Sw}[1]{\begin{sideways}#1\end{sideways}}
\newcommand{\Ast}{*}
\definecolor{lg}{rgb}{.8,.8,.8}
\newcommand{\Dim}{\cellcolor{lg}}
\newcommand{\Tidx}[1]{\ensuremath{_{\mathnormal{#1}}}}
\newcommand{\Rollen}[1]{\ensuremath{\langle}#1\ensuremath{\rangle}}
\newcommand{\Qc}[1]{\texttt{#1}}
\newcommand{\acr}[1]{{#1}}
\newcommand{\ak}[1]{#1\marginpar{\textcolor{textblue}{\footnotesize #1}}}
\newcommand{\mar}[1]{\marginpar{\textcolor{textblue}{\footnotesize #1}}}
\newcommand{\zB}{z.\thinspace{}B.~}
\newcommand{\oA}{o.\thinspace{}ä.~}
\newcommand{\idR}{i.\,d.\,R.\ }
\newcommand{\BBel}[1]{\B[-5]{#1}}
\newcommand{\Bewegtes}[1]{\ensuremath{_{\textrm{#1}}}}
\newcommand{\ORi}{\Bewegtes{1}}
\newcommand{\ORii}{\Bewegtes{2}}
\newcommand{\ORiii}{\Bewegtes{3}}
\newcommand{\ORiv}{\Bewegtes{4}}
\newcommand{\ORv}{\Bewegtes{5}}
\newcommand{\Spur}[1]{t\Sub{#1}}
\newcommand{\Ti}{\Spur{1}}
\newcommand{\Tii}{\Spur{2}}
\newcommand{\Tiii}{\Spur{3}}
\newcommand{\Tiv}{\Spur{4}}
\newcommand{\Akz}{ˈ}
\newcommand{\Nakz}{ˌ}
\newcommand{\PhPr}[1]{\ensuremath{\stackrel{\textnormal{#1\ }}{\Longrightarrow}}}
\newcommand{\phopro}{\ensuremath{\Rightarrow}}

\newcommand{\KTArr}[1]{\ding{226}~\textit{#1}~\ding{226}}

\newcommand{\VfTest}{\KTArr{VfTest}~}
\newcommand{\PronTest}{\KTArr{PronTest}~}
\newcommand{\KoorTest}{\KTArr{KoorTest}~}
\newcommand{\onestar}{◆◇◇}
\newcommand{\twostar}{◆◆◇}
\newcommand{\tristar}{◆◆◆}
\newcommand{\RPr}{\ensuremath{\ll}}
\newcommand{\RUn}{\ensuremath{\sim}}
\newcommand{\REq}{\ensuremath{=}}
\newcommand{\Opsional}{★~}
\newcommand{\Nono}{---}
\newcommand{\Sub}[1]{\ensuremath{_{\text{#1}}}}
\newcommand{\Up}[1]{\ensuremath{^{\text{#1}}}}
\newcommand{\UpSub}[2]{\ensuremath{^{\text{#1}}_{\text{#2}}}}
\newcommand{\TuBegin}{\ding{217}}
\newcommand{\TuEnd}{}
\newcommand{\Folgt}{\ding{217}}

\newcommand*\circlearound[1]{\tikz[baseline=(char.base)]{\node[shape=circle,draw,inner sep=2pt] (char) {#1};}}

\newcommand{\Mark}[2][]{\textbf{#2}}
\newcommand{\EGBDRef}[1]{EGBD, Kapitel~\textit{#1}}


% boxes and stuff for definitions, axioms etc.
\definecolor{textblue}{rgb}{0,0,.5}
\definecolor{textred}{rgb}{.5,0,0}
\definecolor{textgreen}{rgb}{0,.5,0}
\definecolor{lightblue}{rgb}{.9,.9,1}
\definecolor{lightgreen}{rgb}{.9,1,.9}
\definecolor{lightred}{rgb}{1,.9,.9}
\definecolor{lightyellow}{rgb}{1,1,.8}
\definecolor{lightgray}{rgb}{.88,.88,.88}
\definecolor{lsLightgray}{gray}{0.88}
\definecolor{lsYellow}{cmyk}{0,0.25,1,0}

\newcommand{\whyte}[1]{\textcolor{white}{#1}}

% bookkeeping of own environments for definitions etc.
\newlistof[chapter]{dgdef}{dgd}{Verzeichnis der Definitionen}
\newlistof[chapter]{dgsatz}{dgs}{Verzeichnis der Sätze}
\newlistof[chapter]{dguebung}{dga}{Verzeichnis der Übungen}
\newlistof{dgtext}{dgt}{Verzeichnis der Texte}
\newlistof[chapter]{dgvertief}{dgv}{Verzeichnis der Vertiefungen}

\newcounter{deskgram-ziel}
\newcounter{deskgram-ex}
\newcounter{deskgram-strsch}
\newcounter{deskgram-wfilt}
\newcounter{deskgram-pholproz}


% Fix spacing between numbers and caption in list of figures etc.
\makeatletter
  \renewcommand*\l@figure{\@dottedtocline{1}{1em}{3.2em}}
  \renewcommand*\l@table{\@dottedtocline{1}{1em}{3.2em}}
  \renewcommand*\l@dgdef{\@dottedtocline{1}{1em}{3.2em}}
  \renewcommand*\l@dgsatz{\@dottedtocline{1}{1em}{3.2em}}
  \renewcommand*\l@dguebung{\@dottedtocline{1}{1em}{3.2em}}
  \renewcommand*\l@dgtext{\@dottedtocline{1}{1em}{3.2em}}
\makeatother


% NEW Vertiefung (non-floating)

\newcommand{\tblsthickline}{{\color{gray}\rule{\textwidth}{1.5mm}}}

\newenvironment{Sandwich}[1]
  {
    \par\vspace{5mm}\noindent\tblsthickline
    {\par\vspace{3mm}\noindent\sffamily\large\bfseries#1\vspace{4mm}}
  }
  {\par\vspace{3mm}\noindent\tblsthickline\par\vspace{5mm}}


\newenvironment{Vertiefung}[1]
  { 
    \refstepcounter{dgvertief}
    \begin{Sandwich}{#1\hfill Vertiefung~\thechapter.\arabic{dgvertief}}
  }
  {\end{Sandwich}}


% if not using hyperref, redefine this as empty:
\newcommand{\Phantom}{\phantomsection}


\newcommand{\Definition}[2]{
  \refstepcounter{dgdef}
  \tblscolorbox[report]{lsLightGray}{#1\hfill Definition~\thechapter.\arabic{dgdef}}{#2}
}



\newcommand{\Satz}[2]{
  \refstepcounter{dgsatz}
  \tblscolorbox[glass]{lsLightGray}{#1\hfill Satz~\thechapter.\arabic{dgsatz}}{#2}
}

\newcommand{\Uebung}[2]{
  \refstepcounter{dguebung}
  \tblscolorbox[test]{lsLightGray}{#1\hfill Übung~\thechapter.\arabic{dguebung}}{#2}
}

\newcommand{\Merke}[2]{
  \tblscolorbox[bulb]{lsLightGray}{#1}{#2}
}


\newcommand{\Text}[2]{
  \refstepcounter{dgtext}
  \tblscolorbox[book]{lsLightGray}{#1\hfill Text~\arabic{dgtext}}{#2}
}


\newcommand{\Voraussetzungen}[1]{
  \tblsframebox[refresh]{lsYellow}{Voraussetzungen für dieses Kapitel}{#1}
}


\newcommand{\Zusammenfassung}[1]{
  \tblsframebox[refresh]{lsYellow}{Zusammenfassung von Abschnitt~\thesection}{#1}
}



\newcommand{\WFiltTree}[7][0mm]{%
  \refstepcounter{deskgram-wfilt}
  \tblscolorbox[filter]{lsLightGray}{#2\hfill Wortklassenfilter~\arabic{deskgram-wfilt}}{
    \label{#3}
    \hspace{-5pt}\centering
    \begin{tikzpicture}[baseline]
    \node at (0,0) (Wort) [align=left] {#4};
    \node [right=of Wort.east, text width=3.5cm, align=left] (Filter) {#5};
    \node [above right=\baselineskip and 1cm of Filter.east] (Ja) {#6};
    \node [below right=\baselineskip and 1cm of Filter.east] (Nein) {#7}; 
    \path (Wort) edge [-{Latex[round]}] (Filter);
    \path (Filter.east) edge [-{Latex[round]}] node [above,sloped] {Ja} (Ja.west);
    \path (Filter.east) edge [-{Latex[round]}] node [below,sloped] {Nein} (Nein.west);
    \aeundefinethesenodes{Wort, Filter, Ja, Nein}
  \end{tikzpicture}\vspace{#1}%
  }
}

\newcommand{\strschemspace}{\hspace{1em}}
\newcommand{\Phrasenschema}[2]{
  \refstepcounter{deskgram-strsch}
  \tblscolorbox[tree]{lsLightGray}{#1\hfill Phrasenschema~\arabic{deskgram-strsch}}{
    \centering
    #2
  }
}



% OLD "further reading"
\newcommand{\WeitereLiteratur}[1]{
  \tblsframebox[book]{lsLightGray}{Weiterführende Literatur zu \thepart}{#1}
}


% Excercises.
\newcounter{Exer}[chapter]


\renewcommand{\Phantom}{}


%%%%%%%%%%%%%%%%%%%%%%%%%%%%%%%%%%%%%%
% Text used in more than one chapter %
%%%%%%%%%%%%%%%%%%%%%%%%%%%%%%%%%%%%%%

\newcommand{\DefWort}{Das (\textit{lexikalische}) \textit{Wort} ist eine Repräsentation von paradigmatisch zusammengehörenden Wortformen.
Umgangssprachlich kann man von der Zusammenfassung aller möglichen Formen eines Wortes sprechen.
Für das lexikalische Wort sind die Werte nur für diejenigen Merkmale spezifiziert, die in allen Wortformen des Paradigmas dieselben Werte haben.
Die restlichen Werte werden gemäß der Position im Paradigma bei den konkret vorkommenden Wortformen des Wortes gesetzt.}

\newcommand{\ThePhrasenExOne}{dass [[dem Jungen] [die Mutter] [ein Eis] [geschenkt hat]]}
\newcommand{\ThePhrasenExTwo}{dass [[ein Eis] [die Mutter] [dem Jungen] [geschenkt hat]]}


%%%%%%%%%%%%
%%% TikZ %%%
%%%%%%%%%%%%

% Undefining node names.

\makeatletter

\long\def\ifnodedefined#1#2#3{%
  \@ifundefined{pgf@sh@ns@#1}{#3}{#2}}

\newcommand\aeundefinenode[1]{%
  \expandafter\ifx\csname pgf@sh@ns@#1\endcsname\relax
  \else
    \typeout{Undefining node "#1"}%
    \global\expandafter\let\csname pgf@sh@ns@#1\endcsname\relax
  \fi
}

\newcommand\aeundefinethesenodes[1]{%
  \foreach \myn  in {#1}
    {%
      \ifnodedefined{\myn}{%
      \expandafter\aeundefinenode\expandafter{\myn}%
    }{}
    }%
}

\newcommand\aeundefinenumericnodes{%
  \foreach \myn in {1,2,...,50}
    {%
      \ifnodedefined{\myn}{%
      \expandafter\aeundefinenode\expandafter{\myn}%
    }{}
    }%
}
\makeatother

% Drawing sonority diagrams.

\newcommand{\plo}{0}
\newcommand{\fri}{0.5}
\newcommand{\nas}{1}
\newcommand{\liq}{1.5}
\newcommand{\vok}{2}

% Save text.
\newcommand{\lastsaved}{}
\newcommand{\textsave}[1]{\gdef\lastsaved{#1}#1}

\newcommand{\SonDiag}[2][0]{%
  \begin{tikzpicture}
    \textsave{.}
    \tikzset{
      normalseg/.style={fill=white},
      extrasyll/.style={circle, draw, fill=white},
      sylljoint/.style={diamond, draw, fill=white}
    }
    \node at (0,\plo) {P};
    \node at (0,\fri) {F};
    \node at (0,\nas) {N};
    \node at (0,\liq) {L};
    \node at (0,\vok) {V};

    % Draw the helper lines if required.
    \ifthenelse{\equal{#1}{0}}{}{%
      \foreach \y in {\plo, \fri, \nas, \liq,\vok} {%
	\draw [dotted, |-|] (0.25, \y) -- (#1.75, \y);
      }
    }

    \foreach [count=\x from 1, remember=\x as \lastx] \p / \y / \g in #2 {
      \ifthenelse{\equal{\y}{-1}}{\textsave{.}}{%

	% Draw the node, either plain, as Silbenbgelenk, or as extrasyllabic.
        \ifthenelse{\equal{\g}{1}}{%
	  \node (\x) [sylljoint] at (\x, \y) {\p};
	}{%
	  \ifthenelse{\equal{\g}{2}}{%
	    \node (\x) [extrasyll] at (\x, \y) {\p};
	  }{%
	    \node (\x) [normalseg] at (\x, \y) {\p};
	  }
	}

	% Draw the connection unless the previous node was not or was empty.
	\ifthenelse{\NOT\equal{\lastsaved}{.}}{%
	  \draw [->] (\lastx) to (\x);
	}{}
	\textsave{1}
      }
    }
    \aeundefinenumericnodes
  \end{tikzpicture}
}

\forestset{
  Ephr/.style={draw, ellipse, thick, inner sep=2pt},
  Eobl/.style={draw, rounded corners, inner sep=5pt},
  Eopt/.style={draw, rounded corners, densely dashed, inner sep=5pt},
  Erec/.style={draw, rounded corners, double, inner sep=5pt},
  Eoptrec/.style={draw, rounded corners, densely dashed, double, inner sep=5pt},
  Ehd/.style={rounded corners, fill=gray, inner sep=5pt,
    delay={content=\whyte{##1}}
  },
  Emult/.style={for children={no edge}, for tree={l sep=0pt}},
  phrasenschema/.style={for tree={l sep=2em, s sep=2em}},
  decide/.style={draw, chamfered rectangle, inner sep=2pt},
  finall/.style={rounded corners, fill=gray, text=white},
  intrme/.style={draw, rounded corners},
  yes/.style={edge label={node[near end, above, sloped, font=\scriptsize]{Ja}}},
  no/.style={edge label={node[near end, above, sloped, font=\scriptsize]{Nein}}},
  sake/.style={tier=preterminal},
  ake/.style={
    tier=preterminal
    },
}



%%%%%%%%%%
% forest %
%%%%%%%%%%

\forestset{
  narroof/.style={roof, inner xsep=-0.25em, rounded corners},
}
